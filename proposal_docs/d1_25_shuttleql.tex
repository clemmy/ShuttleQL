\documentclass{article}

\usepackage[utf8]{inputenc}
\usepackage{enumitem}
\usepackage{graphicx}
\usepackage{tikz, amsmath, amssymb, gensymb}
\usepackage[margin=1in]{geometry}

\title{Project Proposal}
\author{SE464}
\date{\today}

\begin{document}

\begin{titlepage}
\newcommand{\HRule}{\rule{\linewidth}{0.5mm}}

\center

\textsc{\huge University of Waterloo}\\[3cm]
\textsc{\LARGE SE464}\\[1.5cm]
\textsc{\Large Section 001}\\[1.5cm]

\HRule \\[0.75cm]
{ \Huge \bfseries Project Proposal}\\[0.5cm]
\HRule \\[2cm]

\Large Group 25 \\  [8cm]

{\Large \today}\\

\vfill
\end{titlepage}

\noindent\textbf{Project Title:} ShuttleQL (Shuttle Queueing Logistics) \\
\textbf{Team Members:}
\begin{itemize}
  \item Cheng Dong (c9dong)
  \item Zhaotian Fang (z23fang)
  \item Clement Hoang (c8hoang)
  \item Di Sen Lu (dslu)
\end{itemize}

\section{Introduction}
\subsection{Motivation}
Around the world, there are 100s of recreational badminton clubs which host
regular sessions at set locations for members to drop in and play at.
Traditionally, the operation of a badminton club was done manually through the
combined efforts of the clubs executive staff. There are several burdenous
and repetitive tasks that the staff members need to do on a regular basis in
order to keep the club running smoothly. For example,
\begin{itemize}
  \item Registering new club members
  \item Checking in/out members during a club session
  \item Scheduling members into courts for each rotation
\end{itemize}
These tasks negatively impact the productivity of the staff and are prone
to human error.
In addition, there is no effective means for the club execs to communicate
with the members in real-time.
Finally, there is no easy way to get feedback on player performance unless
they hire a coach or ask for critique from a staff member.

\subsection{Idea}
ShuttleQL is an all-in-one badminton club management platform which not only
automates repetitive administrative tasks but also acts as a communication hub
between the club executives and its members. It also offers insights and
analytics into performance of the players in the club.

The platform will be entirely web-based. It will consist of two components:
a club member mobile web based dashboard and an admin desktop based dashboard.
The platform from the club member's perspective will be a mobile web based app
since they are more likely to have access to a phone than a laptop when they
come to session. The admin dashboard will be desktop based since it requires
heavy user interaction which will be easier on a laptop rather than on a phone.

\subsection{Originality}
Although existing software solutions exist for a similar purpose, they are often
incomplete relative to ShuttleQL and don't leverage many of todays technologies.
ShuttleQL aims to satisfy most if not all of the pain points associated with
managing a badminton club. It also applies new concepts such as data analytics and
machine learning into processes such as matchmaking and coaching, which has not
yet been done in existing solutions.

\section{Project Properties}
\subsection{Functional}
\subsubsection{User Scenarios}
\subsection{Non-functional}

\section{Mockups}

\end{document}
